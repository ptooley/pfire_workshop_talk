\pdfminorversion=4
\PassOptionsToPackage{x11names}{xcolor}
\documentclass{beamer}
%\documentclass[handout]{beamer}
%\usepackage{handoutWithNotes}
%\pgfpagesuselayout{4 on 1 with notes}[a4paper, border shrink=5mm]
%% Method from https://github.com/gdiepen/latexbeamer-handoutWithNotes
\usepackage{amsmath}
\usepackage{subcaption}
\usepackage{appendixnumberbeamer}
\usepackage{animate}
\usepackage{grffile}
\usepackage[utf8]{inputenc}
%\usepackage[hang,flushmargin]{footmisc}

\usepackage{tikz}
\usetikzlibrary{shapes.geometric, arrows, positioning}

\usepackage[style=authortitle,backend=biber]{biblatex}
\addbibresource{refs.bib}

\usetheme{Sheffield}
\useinnertheme{SheffieldCBM}

\addbibresource{library.bib}

\newsavebox{\foobox}
\newcommand{\slantbox}[2][0]{\mbox{%
        \sbox{\foobox}{#2}%
        \hskip\wd\foobox
        \pdfsave
        \pdfsetmatrix{1 0 #1 1}%
        \llap{\usebox{\foobox}}%
        \pdfrestore
}}
\newcommand\unslant[2][-.25]{\slantbox[#1]{$#2$}}

%  \item specially for beamer
\renewcommand{\footnotesize}{\scriptsize}
\renewcommand{\vec}{\mathbf}

\tikzset{
    userblock/.style={
    rectangle,
    rounded corners,
    minimum width=3cm,
    minimum height=1cm,
    align = center,
    draw=shefblue,
    line width = 1pt,
  },
  flowblock/.style={
    rectangle,
    rounded corners,
    minimum width=3cm,
    minimum height=1cm,
    align = center,
    draw=sgreylight,
    line width = 1pt,
  },
  arrow/.style={
    line width = 1.5pt,
    ->,
    > = stealth,
    draw=shefblue,
    rounded corners
    },
  dashdouble/.style={
    line width = 1.5pt,
    <->,
    > = stealth,
    draw=shefblue!70,
    },
  dashbend/.style={
    line width = 1.5pt,
    ->,
    > = stealth,
    draw=shefblue!70,
    bend right=20,
    },
}

\newenvironment<>{varblock}[2][.9\textwidth]{%
  \setlength{\textwidth}{#1}
  \begin{actionenv}#3%
    \def\insertblocktitle{#2}%
    \par%
    \usebeamertemplate{block begin}%
}{%
  \par%
    \usebeamertemplate{block end}%
  \end{actionenv}%
}

\begin{document}

\title{Image Registration with pFIRE}
\subtitle{CompBioMed Winter School 2019}
\author[phil.tooley@sheffield.ac.uk]{Phil Tooley\\University of Sheffield}
\date[2019/02/14]{\\14 February 2019}


\begin{frame}
\maketitle
\end{frame}

\section{Introductory Talk 09:00 -- 09:15}
\begin{frame}
  \frametitle{Session Layout}
  \tableofcontents
\end{frame}

\subsection{What is image registration?}
\begin{frame}
  \frametitle{What is image registration?}
  \begin{block}{Image Registration}
    Image registration is the process by which one image is transformed by displacing pixels to
    match another image as closely as possible
  \end{block}
  \begin{block}{}
    The image which is transformed is known as the \emph{moved} image, and the target image to
    which it is matched is known as the \emph{fixed image}
  \end{block}
\end{frame}

\begin{frame}
  \frametitle{What is image registration?}
  \begin{center}
    \includegraphics[scale=0.1]{registration_overlay}
  \end{center}
\end{frame}

\begin{frame}
  \frametitle{Types of registration}
  \only<1>{
    \begin{block}{Rigid Registration}
      \begin{itemize}
        \item Global transformations which affect the whole image in the same way
        \item E.g translation, scaling, linear shearing and rotation
        \item Cannot describe non-linear or local deformations
        \item Relatively simple to compute
        \item Commonly used for alignment
      \end{itemize}
    \end{block}
  }
  \only<2>{
    \begin{block}{Elastic registration}
      \begin{itemize}
        \item Displacement field which describes how individual image pixels are moved
        \item Map can be arbitrarily high resolution 
        \item In principle, describe any transformation
        \item Computationally intensive
        \item Useful in a wide range of applications
      \end{itemize}
      \alert{pFIRE performs \textbf{elastic} registration}
    \end{block}
  }
\end{frame}

\subsection{Applications for elastic registration}
\begin{frame}
  \frametitle{Applications for elastic registration}
  \begin{itemize}
    \item Segmentation by registration
    \begin{itemize}
      \item Manually segment first image
      \item Register other images to first image
      \item Use results to map segmentation to other images
    \end{itemize}
    \item Patient specific modelling (PSM)
    \begin{itemize}
      \item Create a generic model of structure
      \item Register patient image to generic model
      \item Use result to transform generic $\to$ patient specific model
      \item Apply analysis to new PSM
    \end{itemize}
    \item Digital Volume Correlation (Stress/strain analysis)
      \begin{itemize}
        \item Image unstressed sample ($\mu$-CT)
        \item Apply stress and image again
        \item Register to produce displacement field
        \item Differentiate displacement to get strain field in sample
      \end{itemize}
  \end{itemize}
\end{frame}

\subsection{How pFIRE works}
\begin{frame}
  \frametitle{How pFIRE works (briefly!)}
  \begin{itemize}
    \item Represent fixed and moved images with functions $f(\vec{x})$, $m(\vec{x})$
    \item These give the value of the pixel at the point $\vec{x}$
    \item A perfect mapping is a vector field $\vec{R}(\vec{x})$, such that
    \begin{equation}
      m(\vec{x} + \vec{R}(\vec{x})) = f(\vec{x})
    \end{equation}
    \item To register the images, find a function $\vec{R}(\vec{x})$ that satisfies this equation
  \end{itemize}
\end{frame}

\begin{frame}
  \frametitle{Finding the displacement field}
  \begin{itemize}
    \item Taylor expand to get an expression that uses $f(\vec{x})$ and $m(\vec{x})$ 
    \begin{equation}
      f(\vec{x}) - m(\vec{x}) \simeq \frac{1}{2} \vec{R}(\vec{x})\cdot(\nabla \cdot f(\vec{x}) + m(\vec{x}))
    \end{equation}
    \item This now uses the two input images and the gradient of the images
    \item[~]
    \item Problem: equation is underdetermined!
  \end{itemize}
\end{frame}

\begin{frame}
  \frametitle{Finding the displacement field}
  \begin{center}
    \includegraphics[scale=0.1]{registration_overlay}
  \end{center}
\end{frame}

\begin{frame}
  \frametitle{Finding the displacement field}
  \begin{itemize}
    \item Need to add additional constraints (Tikhonov Regularisation)
    \item Constrain allows us to choose solution $\to$ want smoothest solution
    \item Displacement is smooth if second derivative is zero everywhere
    \item Laplacian operator:
    \begin{equation}
      \nabla^2 \vec{R}(\vec{x}) = 0
    \end{equation}
  \end{itemize}
\end{frame}

\begin{frame}
  \frametitle{Finding the displacement field}
  \begin{itemize}
    \item Problem is actually discrete -- have pixels in image and nodes in displacement field
    \item Typically $N_{pixels} \gg N_{nodes}$, problem has $N_{nodes}$ DOF
    \item Each map node uses information from many pixels
    \item Represent as a matrix equation
    \begin{equation}
      \begin{bmatrix}\vec{f}-\vec{m} \\ 0\end{bmatrix} =
        \begin{bmatrix}\vec{T} \\ \vec{L}\end{bmatrix} \cdot \vec{R}
    \end{equation}
    \item $\vec{T}$ is a non-square matrix mapping pixel information to map node terms
    \item $\vec{L}$ is the Laplacian matrix
  \end{itemize}
\end{frame}

\begin{frame}
  \frametitle{The pFIRE Algorithm}
  \begin{itemize}
    \item pFIRE solves the matrix equation:
    \begin{equation}
      \begin{bmatrix}\vec{T}^t & \vec{L}^t\end{bmatrix} \cdot
      \begin{bmatrix}\vec{f}-\vec{m} \\ 0\end{bmatrix} =
        \begin{bmatrix}\vec{T}^t \vec{T} + \vec{L}^t \vec{L} \end{bmatrix} \cdot \vec{R}
    \end{equation}
    \item Iterative solution: solve the matrix equation many times
    \item Start with coarse node spacing
    \item Each iteration converges towards the solution
    \item Step converged when displacement field stops changing
    \item Increasingly refine node spacing when previous step converges
  \end{itemize}
\end{frame}

\begin{frame}
  \frametitle{The pFIRE Algorithm}
  \begin{center}
    \begin{tikzpicture}
      \node (start) [flowblock] {Start with 3\\nodes per dim};
      \node (innerstart) [flowblock, right = 1cm of start]{Solve equation \&\\Update displacements};
      \draw [arrow] (start) -> (innerstart);
      \node (innerbranch) [flowblock, right = 1cm of innerstart]{Check for\\convergence};
      \draw [arrow] (innerstart) -- (innerbranch);
      \draw [arrow] (innerbranch) |- node [below, pos=0.8]{Not converged} ($ (innerstart) + (0.5,1.5) $) 
                                  -| (innerstart);
      \node (outerbranch) [flowblock, below = 1cm of innerstart]{Refine node\\ spacing if needed};
      \draw [arrow] (innerbranch) |- node[above, pos=0.7]{Converged}(outerbranch);
      \draw [arrow] (outerbranch) -- (innerstart);
      \node (finish) [flowblock, below = 1cm of start] {Finish};
      \draw [arrow] (outerbranch) -- (finish);
    \end{tikzpicture}
  \end{center}
  \alert Detailed algorithm description in Barber and Hose (2005) or Barber et al. (2007)
\end{frame}

\section{Hands on tutorial: 09:15 -- 10:45}
\subsection{Access pFIRE on MareNostrum}
\subsection{How to run pFIRE}
\subsection{Visualising the output}

\begin{frame}
  \frametitle{Hands-on tutorial session}
  \begin{itemize}
    \item pFIRE Tutorial: https://insigneo.github.io/pFIRE/docs.html
    \item[~]
    \item MareNostrum readme: /gpfs/projects/nct00/nct00016/README
    \item Follow the instructions to set up pFIRE
    \item[~]
    \item MareNostrum files: /gpfs/projects/nct00/nct00016/tutorial\_files
    \item Copy this to your own account
    \item[~]
    \item Please complete the application survey
  \end{itemize}
\end{frame}

\section{Coffee Break: 10:45 -- 11:00}
\begin{frame}
\end{frame}

\end{document}
